
\section*{Assignment 1}

\maketitle

\begin{enumerate}

%%%%%%%%%%%%%%%%%%%%%%%%%%%%%%%%%%%%%%%%%%%%%%%%%%%%%%%%%%%%%%%%%%%%%%%%%%%%%%%%%%%%%%%
%%%%%%%%%%%%%%%%%%%%%%%%%%%%%%%%%%%%%%%%%%%%%%%%%%%%%%%%%%%%%%%%%%%%%%%%%%%%%%%%%%%%%%%
\item \textbf{(5)} Consider the matrix $H(v)=\hat 1 - 2 v v^T$, where $v$ is a unit column vector. What is the rank of the matrix $H(v)$? Prove that it is orthogonal.
%%%%%%%%%%%%%%%%%%%%%%%%%%%%%%%%%%%%%%%%%%%%%%%%%%%%%%%%%%%%%%%%%%%%%%%%%%%%%%%%%%%%%%%
%%%%%%%%%%%%%%%%%%%%%%%%%%%%%%%%%%%%%%%%%%%%%%%%%%%%%%%%%%%%%%%%%%%%%%%%%%%%%%%%%%%%%%

\begin{solution}
    Let's prove that $A = A^{T}$
    \begin{eqnarray*}
        A^{T} = \left(I_{n} - 2 v v^{T}\right)^{T} = I_{n} - 2 (v v^{T})^{T} =  I_{n} - 2 v v^{T}
    \end{eqnarray*}

    So $A^{T} = A$. Calculating $AA^{T}$ 
    \begin{eqnarray*}
        AA^{T} = A^{2} = \left(I_{n} - 2 v v^{T}\right)\left(I_{n} - 2 v v^{T}\right) = I_{n} - 4vv^{T} + 4vv^{T}vv^{T} = \lvert v^{T} v = 1\rvert  = I_{n}
    \end{eqnarray*} 
    from which we conclude that $A$ is both orthogonal and indepotent matrix.

    For indepotent matrices following statement is true: 
    \begin{eqnarray*}
        \tr A  = \rank A \\
        \tr A = n - 2v^{T}v =  n - 2 \\
        \rank A = n - 2
    \end{eqnarray*}
\end{solution}

\begin{solution}
    \begin{equation*}
        H v = - v
    \end{equation*}
    So $v$ is an eigenvector with eigenvalue $\lambda = - 1$. 
    If we take any another vector $u$ such that $u \perp v$, we get $H u = u$. So there are only two eigenvalues $\lambda_{1} = -1$ and $\lambda_{2} = 1$. 
    It means that there exists some transformation $V$ which makes $H$ diagonal matrix $\operatorname*{diag}\left(- 1, 1, \dots 1\right)$. This matrix is it's own inverse
    hence orthogonal. Beside that it is indepotent matrix so $\rank A = n - 2$
\end{solution}


%%%%%%%%%%%%%%%%%%%%%%%%%%%%%%%%%%%%%%%%%%%%%%%%%%%%%%%%%%%%%%%%%%%%%%%%%%%%%%%%%%%%%%%
%%%%%%%%%%%%%%%%%%%%%%%%%%%%%%%%%%%%%%%%%%%%%%%%%%%%%%%%%%%%%%%%%%%%%%%%%%%%%%%%%%%%%%%
\item \textbf{(10)} Prove the following inequalities and provide examples of $x$ and $A$ when they turn into equalities:
\begin{itemize}
\item $\norm{x}_2 \le \sqrt{m}\norm{x}_\infty$
\item $\
\norm{A}_\infty \le \sqrt{n} \norm{A}_2$
\end{itemize}
where $x$ is a vector of $m$ components and $A$ is $m\times n$ matrix.
%%%%%%%%%%%%%%%%%%%%%%%%%%%%%%%%%%%%%%%%%%%%%%%%%%%%%%%%%%%%%%%%%%%%%%%%%%%%%%%%%%%%%%%
%%%%%%%%%%%%%%%%%%%%%%%%%%%%%%%%%%%%%%%%%%%%%%%%%%%%%%%%%%%%%%%%%%%%%%%%%%%%%%%%%%%%%%%

\begin{solution}
    \begin{itemize}
        \item \begin{eqnarray*}
            |x_{i}| \leq \sup_{j} |x_{j}| := \Vert x\Vert_{\infty} \Rightarrow x^2_{i} \leq \left\lVert x\right\rVert^{2}_{\infty} \\
            \Vert x \Vert_{2}  = \sqrt{\sum_{i}^{m} x^{2}_i} \leq  \sqrt{\sum_{i}^{m} \Vert x \Vert_{\infty}^2} = \sqrt{m \Vert x \Vert_{\infty}^2 } =  \sqrt{m}\Vert x \Vert_{\infty}
        \end{eqnarray*}

        \item \begin{eqnarray*}
            \Vert A \Vert_{2} = \sup_{x \ne 0} \frac{\Vert A x \Vert_2}{\Vert x \Vert_2} 
            \leq \sup_{x \ne 0} \frac{\sqrt{m}\Vert A x \Vert_{\infty}}{\Vert x \Vert_2}
            \leq \frac{\sqrt{m}\Vert A x \Vert_{\infty}}{\Vert x \Vert_{\infty}}
            =  \sqrt{m} \Vert A \Vert_{\infty}
        \end{eqnarray*}

        \begin{eqnarray*}
            \Vert Ax\Vert_\infty\leq\Vert Ax\Vert_2\leq\Vert A\Vert_2\Vert x\Vert_2\leq\Vert A\Vert_2\sqrt{n}\Vert x\Vert_\infty \Rightarrow \norm{A}_\infty \le \sqrt{n} \norm{A}_2
        \end{eqnarray*}

        \begin{eqnarray*}
            \norm{x}_{\infty} \leq \norm{x}_{2} \leq \sqrt{m} \norm{x}_{\infty} \\
            \norm{A}_{2} = \frac{\norm{Ax}_{2}}{\norm{x}_{2}} \leq \frac{\sqrt{n} \norm{Ax}_{\infty}}{\norm{x}_{\infty}} =  \sqrt{n} A_{\infty} \\
            \norm{A}_{\infty} = \frac{\norm{Ax}_{\infty}}{\norm{x}_{\infty}} \leq \frac{\sqrt{m} \norm{Ax}_{2}}{\norm{x}_{2}} =  \sqrt{m} A_{2} \\
        \end{eqnarray*}
    \end{itemize}
\end{solution}


%%%%%%%%%%%%%%%%%%%%%%%%%%%%%%%%%%%%%%%%%%%%%%%%%%%%%%%%%%%%%%%%%%%%%%%%%%%%%%%%%%%%%%%
%%%%%%%%%%%%%%%%%%%%%%%%%%%%%%%%%%%%%%%%%%%%%%%%%%%%%%%%%%%%%%%%%%%%%%%%%%%%%%%%%%%%%%%
\item \textbf{(5)} Assuming $u$ and $v$ are $m$-vectors, consider the matrix $A=1+u v^T$ which is a rank-one perturbation of identity. Can it be singular? Assuming it is not, compute its inverse. You may look for it in a form of $A^{-1}=1+\alpha u v^T$ for some scalar $\alpha$ and evaluate $\alpha$.
%%%%%%%%%%%%%%%%%%%%%%%%%%%%%%%%%%%%%%%%%%%%%%%%%%%%%%%%%%%%%%%%%%%%%%%%%%%%%%%%%%%%%%%
%%%%%%%%%%%%%%%%%%%%%%%%%%%%%%%%%%%%%%%%%%%%%%%%%%%%%%%%%%%%%%%%%%%%%%%%%%%%%%%%%%%%%%%

%%%%%%%%%%%%%%%%%%%%%%%%%%%%%%%%%%%%%%%%%%%%%%%%%%%%%%%%%%%%%%%%%%%%%%%%%%%%%%%%%%%%%%%
%%%%%%%%%%%%%%%%%%%%%%%%%%%%%%%%%%%%%%%%%%%%%%%%%%%%%%%%%%%%%%%%%%%%%%%%%%%%%%%%%%%%%%%
\item \textbf{(5)} Prove that for any unitary matrix $U$ one has $\norm{UA}_F=\norm{AU}_F=\norm{A}_F$.
%%%%%%%%%%%%%%%%%%%%%%%%%%%%%%%%%%%%%%%%%%%%%%%%%%%%%%%%%%%%%%%%%%%%%%%%%%%%%%%%%%%%%%%
%%%%%%%%%%%%%%%%%%%%%%%%%%%%%%%%%%%%%%%%%%%%%%%%%%%%%%%%%%%%%%%%%%%%%%%%%%%%%%%%%%%%%%%

%%%%%%%%%%%%%%%%%%%%%%%%%%%%%%%%%%%%%%%%%%%%%%%%%%%%%%%%%%%%%%%%%%%%%%%%%%%%%%%%%%%%%%%
%%%%%%%%%%%%%%%%%%%%%%%%%%%%%%%%%%%%%%%%%%%%%%%%%%%%%%%%%%%%%%%%%%%%%%%%%%%%%%%%%%%%%%%
\begin{comment}
\item \textbf{(5)} In this exercise, you are expected to use vectorized NumPy operations as much as possible.
     \begin{itemize}
         \item Generate a random matrix:
\lstset{language=Python}
\lstset{frame=lines}
\lstset{label={lst:code_direct}}
\lstset{basicstyle=\ttfamily}
\begin{lstlisting}
n = 100
A = np.random.normal(size=(n, n))
\end{lstlisting}
\item Plot the distribution function of the angles $\angle (a_i, a_j)$ where $a_i$ are columns of the matrix $a$ and $i\ne j$. How does the standard deviation of this distribution decay with $n$ (check this empirically)?
\item Construct the matrix $a^\prime$ with column vectors obtained by normalizing the columns of $a$.
\end{itemize}
\end{comment}
%%%%%%%%%%%%%%%%%%%%%%%%%%%%%%%%%%%%%%%%%%%%%%%%%%%%%%%%%%%%%%%%%%%%%%%%%%%%%%%%%%%%%%%
%%%%%%%%%%%%%%%%%%%%%%%%%%%%%%%%%%%%%%%%%%%%%%%%%%%%%%%%%%%%%%%%%%%%%%%%%%%%%%%%%%%%%%%

%%%%%%%%%%%%%%%%%%%%%%%%%%%%%%%%%%%%%%%%%%%%%%%%%%%%%%%%%%%%%%%%%%%%%%%%%%%%%%%%%%%%%%%
%%%%%%%%%%%%%%%%%%%%%%%%%%%%%%%%%%%%%%%%%%%%%%%%%%%%%%%%%%%%%%%%%%%%%%%%%%%%%%%%%%%%%%%
\begin{comment}
\item \textbf{(5)} Implement the function \lstinline{sum_array(A)} which takes a 2D numpy array \lstinline{A} as input and returns the sum of the array's elements. Consider doing it in 3 ways: (i) looping through the array explicitely, (ii) doing the jit version of (i), and (iii) using NumPy's built-in functionality. Time these three approaches for large enough arrays.
\end{comment}
%%%%%%%%%%%%%%%%%%%%%%%%%%%%%%%%%%%%%%%%%%%%%%%%%%%%%%%%%%%%%%%%%%%%%%%%%%%%%%%%%%%%%%%
%%%%%%%%%%%%%%%%%%%%%%%%%%%%%%%%%%%%%%%%%%%%%%%%%%%%%%%%%%%%%%%%%%%%%%%%%%%%%%%%%%%%%%%

%%%%%%%%%%%%%%%%%%%%%%%%%%%%%%%%%%%%%%%%%%%%%%%%%%%%%%%%%%%%%%%%%%%%%%%%%%%%%%%%%%%%%%%
%%%%%%%%%%%%%%%%%%%%%%%%%%%%%%%%%%%%%%%%%%%%%%%%%%%%%%%%%%%%%%%%%%%%%%%%%%%%%%%%%%%%%%%
\item \textbf{(10)} In this exercise your goal will be to study and speed up an implementation of \href{https://en.wikipedia.org/wiki/K-means_clustering}{K-means algorithm}. In the notebook \path{kmeans.ipynb}, you can find a naive implementation. Explore the code, make sure you understand it. You will find there two functions \texttt{dist\_i} and \texttt{dist\_ij} which are (on purpose) implemented in a rather inefficient way. Improve them by getting rid of the loops in the favor of a proper numpy vectorized implementation and measure the speed-up of the full algorithm for $N=10000$.
%%%%%%%%%%%%%%%%%%%%%%%%%%%%%%%%%%%%%%%%%%%%%%%%%%%%%%%%%%%%%%%%%%%%%%%%%%%%%%%%%%%%%%%
%%%%%%%%%%%%%%%%%%%%%%%%%%%%%%%%%%%%%%%%%%%%%%%%%%%%%%%%%%%%%%%%%%%%%%%%%%%%%%%%%%%%%%%

%%%%%%%%%%%%%%%%%%%%%%%%%%%%%%%%%%%%%%%%%%%%%%%%%%%%%%%%%%%%%%%%%%%%%%%%%%%%%%%%%%%%%%%
%%%%%%%%%%%%%%%%%%%%%%%%%%%%%%%%%%%%%%%%%%%%%%%%%%%%%%%%%%%%%%%%%%%%%%%%%%%%%%%%%%%%%%%
\item \textbf{(10)} Some things just can not be vectorized but still can be sped up compared to naive implementation. For example, consider computation of the \href{https://mathworld.wolfram.com/Hofstadter-Conway10000-DollarSequence.html}{Hofstadter-Conway sequence} $a(n)$ such that $a(1)=1$, $a(2)=1$ and
\begin{equation}
a(n) = a(a(n-1)) + a(n-a(n-1)),\quad n > 2
\end{equation}
Write three functions, computing the sequence up to $n$-th element in three ways: i) pre-allocating \texttt{numpy} array and filling it using \texttt{for} loop, ii) cumulatively appending \texttt{python} list and converting it to \texttt{numpy} array, iii) same as i)  but compiled (\texttt{jit}) version. Time the resulting implementations and conclude which is preferable. With the optimal one, compute $a(10^8)$.
%%%%%%%%%%%%%%%%%%%%%%%%%%%%%%%%%%%%%%%%%%%%%%%%%%%%%%%%%%%%%%%%%%%%%%%%%%%%%%%%%%%%%%%
%%%%%%%%%%%%%%%%%%%%%%%%%%%%%%%%%%%%%%%%%%%%%%%%%%%%%%%%%%%%%%%%%%%%%%%%%%%%%%%%%%%%%%%

%%%%%%%%%%%%%%%%%%%%%%%%%%%%%%%%%%%%%%%%%%%%%%%%%%%%%%%%%%%%%%%%%%%%%%%%%%%%%%%%%%%%%%%
%%%%%%%%%%%%%%%%%%%%%%%%%%%%%%%%%%%%%%%%%%%%%%%%%%%%%%%%%%%%%%%%%%%%%%%%%%%%%%%%%%%%%%%
\item \textbf{(15*)} Consider a function mapping six tensors to one tensor: $Z\left(\lambda^{(1)},\lambda^{(2)},\lambda^{(3)},\Gamma^{(1)}, \Gamma^{(2)}, U\right)$, with
\begin{equation}
Z_{ahij} = \sum_{bcdefg}\lambda^{(1)}{}_{ab}\Gamma^{(1)}{}_{cbd}\lambda^{(2)}{}_{de}\Gamma^{(2)}{}_{feg}\lambda^{(3)}{}_{gh}U_{ijcf}\label{eq:TEBD}.
\end{equation}
Assume that all indices of the tensors appearing above take values from $1$ to $\chi$. Running the numerical experiments, explore the values of $\chi$ in the range $3$--$50$ (from slowest to fastest implementation).
\begin{itemize}
    \item In the notebook \path{convolution.ipynb} you may find implemented a \emph{stupid} way to compute this convolution, which takes $\chi^4\times \chi^{6} = \chi^{10}$ flops. In fact, this can be computed much faster!
    \item Using the function \texttt{numpy.einsum} (its crucial to use the \texttt{optimize} argument), you can actually achieve a much faster implementation. In order to understand what it is doing under the hood, explore the function \texttt{numpy.einsum\_path}. What is the minimal number of flops required for computation of $Z$?
    \item Using the understanding of the output of \texttt{numpy.einsum\_path}, implement an algorithm to compute $Z$, which is as effective as \texttt{numpy.einsum}, but relying only on more elementary \texttt{numpy.dot} and \texttt{numpy.tensor\_dot}.
\end{itemize}

%%%%%%%%%%%%%%%%%%%%%%%%%%%%%%%%%%%%%%%%%%%%%%%%%%%%%%%%%%%%%%%%%%%%%%%%%%%%%%%%%%%%%%%
%%%%%%%%%%%%%%%%%%%%%%%%%%%%%%%%%%%%%%%%%%%%%%%%%%%%%%%%%%%%%%%%%%%%%%%%%%%%%%%%%%%%%%%

\end{enumerate}

